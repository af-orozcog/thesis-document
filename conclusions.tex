\chapter{Conclusions and Future Work}
\label{cha:conclusions}

The rising importance of Kotlin is imminent and so is the necessity to develop tools that would ease the work of developers for finding bugs and fixing them. Nonetheless, most of the debuggers for Kotlin are still based on the breakpoint feature. But by using a back-in-time debugger, Kotlin developers would augment the opportunities to understand the reason behind the bugs in their applications, by creating hypothetical scenarios and replaying the execution of the program from a point onwards.

A back-in-time debugger for Kotlin relying on the MVIKotlin framework for its functionality is a possibility that worked and could replicate most of the use cases of DeloreanJS, except for the breakpoint feature. Nonetheless, this debugger still have problems to overcome:

\begin{itemize}
\item \textbf{Usability test}: Given that the back-in-time debugger for Kotlin proposes a different way to debug applications that adds on the necessity of the developer to adhere to a MVI framework, it is important to evaluate the usability of this new method. To check for the usability it is proposed to replicate the methodology done in which is also used to test DeloreanJS. This methodology consists of using an experimental and control group of developers and giving them the task of finding and fixing bugs to test the back-in-time debugger against the built-in Kotlin debugger in Intellij IDEs like Android Studio. This test would also include an evaluation on how hard and convenient it is perceived by the developers that coding using the MVI architectural pattern is for different tasks.
\item \textbf{Rigidness and alternatives for client-server communication}: Currently the back-in-time debugger supports a very limited way of changing the state of a \textbf{store}. The developer needs to define a method that can be called only by an array of either numbers or strings. This would make it very difficult for the coder to for instance receive a matrix to be used in an \textbf{intent} to change a past event. This is why more research on the possible ways to communicate the Intellij plugin with the server must be done in order to ease the previous problem, specially new alternatives to the currently used proto library may be helpful to more intuitively change the state of a \textbf{store} in the past.
\end{itemize}